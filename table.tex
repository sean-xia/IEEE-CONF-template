\section{Table}
An example of a floating table. Note that, for IEEE style tables, the
\verb|\caption| command should come BEFORE the table. Table text will default to
\verb|\footnotesize| as IEEE normally uses this smaller font for tables.
The \verb|\label| must come after \verb|\caption| as always.

\begin{table}
\caption{\textbf{\label{tab:1} An Example of a Table}}
\begin{tabular}{cc}
\hline
One & Two\\
\hline
Three & Four\\
\hline
\end{tabular}
\end{table}


Note that IEEE does not put floats in the very first column - or typically
anywhere on the first page for that matter. Also, in-text middle ("here")
positioning is not used. Most IEEE journals/conferences use top floats
exclusively. Note that, LaTeX2e, unlike IEEE journals/conferences, places
footnotes above bottom floats. This can be corrected via the \verb|\fnbelowfloat|
command of the stfloats package.

